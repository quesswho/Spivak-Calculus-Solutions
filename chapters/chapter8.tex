\chapter{Least Upper Bounds}

\begin{problem}
	If maximum or minimum is not mentioned, then they do not exist
	\enum{
	\item Maximum is $1$, infimum is $0$.
	\item Maximum is $1$, minimum is $-1$.
	\item Maximum is $1$, minimum is $1$.
	\item Minimum is $0$, supremum is $\sqrt{2}$.
	\item Unbounded.
	\item The infimum is $\frac{-1-\sqrt{5}}{2}$ and the supremum is $\frac{-1+\sqrt{5}}{2}$.
	\item The infimum is $\frac{-1-\sqrt{5}}{2}$ and the supremum is $0$.
	\item The infimum is $-1$ and the maximum is $1.5$.
	}
\end{problem}

\begin{problem}
	\enuma{
	\item Let $$-A=\{\, -x : x \in A \, \}\ $$
	Suppose $A\neq \emptyset$, then there must exist an $a\in A$. Thus $-a \in -A$ by definition. \\\\
	Suppose $A$ is bounded below, then there is a $y \in \mathbb{R}$ such that $y\leq x$ for any $x\in A$. Equivalently we have $-y\geq -x$. This means that $-A$ is bounded above. Consequently there is a least upper bound $\sup(-A)=\alpha$ for $-A$. By definition $-x\leq \alpha\leq b$ for any upper bound $b$ for $-A$. It follows that $-b\leq -\alpha \leq x$ since $-b$ is any lower bound it means that $-\alpha=-\sup(-A)$ is the greatest lower bound.
	
	\item Suppose $A$ is a nonempty set that is bounded below. Let $B$ be the set containing every lower bound of $A$. Since $A$ is bounded below, the set $B$ is nonempty. Since $B$ contains every upper bound it is certainly true that for every $x\in A$ and $y \in B$ we have $x\geq y$, that is $B$ is bounded above by every element of $A$. Since $B$ is bounded above there must be a least upper bound $\alpha$ (by property 13). Now it remains to show that $\alpha$ is a greatest lower bound.\\\\
	Since $A$ is a subset of the set of upper bounds for $B$ we have in particular that for any $a\in A$ it is true that $\alpha\leq a$. This means $\alpha$ is a lower bound for A which satisfies (1) for the definition of a greatest lower bound.\\\\
	Now, because $\alpha$ is an upper bound it is certainly true that for any $b \in B$ it follows that $\alpha  \geq b$. Since $B$ contains every upper bound for $A$ this means (2) is satisfied. Thus we have shown that $\sup{B}=\alpha=\inf{A}$.
	}
\end{problem}

\begin{problem}
	\enuma{
	\item There is not necessarily a second smallest $x$, consider a function which intersects the horizontal axis once, then there is only one $x$ satisfying $f(x)=0$. \\\\
	
	}
\end{problem}