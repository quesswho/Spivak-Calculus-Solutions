\chapter{Continuous Functions}

\begin{problem}
	\enum{
	\item The function has a discontinuity at $x=2$, the limit as $x \to 2$ is $4$ so we define
	\[
	F(x)=
	\begin{cases}
		\frac{x^2-4}{x-2}, & \text{for } x\neq 2\\
		4, & \text{otherwise.}
	\end{cases}
	\]
	
	\item Since the limit at $x=0$ does not even exist, there is nothing we can do here.
	\item $F(x)=0$
	\item The limit at every rational $x$ is zero as proven earlier in the book. So $f$ is discontinuous everywhere. The answer is no.
	}
\end{problem}

\begin{problem}
\end{problem}

\begin{problem}
	\enuma{
	\item Suppose $|f(x)|\leq|x|$, if we let $x=0$ then $|f(0)|\leq0$ and by noting that the absolute value has the property $|f(0)|\geq 0$ we must in fact have $f(0)=0$. Now let $\delta=\varepsilon$ then for every $\epsilon>0$ and any $x$, if $$|x-0|<\delta$$ then $$|f(x)|\leq|x|<\varepsilon$$ which is the same as $$|f(x)-f(0)|<\varepsilon.$$ Thus $f(x)$ is continuous at 0.
	
	\item $f:\emptyset \to \emptyset$
	\item By letting $x=0$ we find that by using similar arguments to a) that $f(0)=0$. For every $\varepsilon>0$ there exists a $\delta>0$ such that for every $x$ if $$|x-a|<\delta$$ then, $$|g(x)|<\varepsilon$$ and then we certainly have $|f(x)|<\varepsilon$ and consequently $$|f(x)-f(0)|<\varepsilon,$$ so $f$ is continuous at zero.
	}
\end{problem}

\begin{problem}
	Let
	\[f(x)=
	\begin{cases}
		1, &\text{x irrational} \\
		-1, & \text{x rational.} 
	\end{cases}
	\]
	The limit of $f(x)$ does not exist, but the function $|f(x)|$ is a constant function, thus it is continuous.
\end{problem}

\begin{problem}
	\[g(x)=
	\begin{cases}
		0, &\text{x irrational}\\
		x, &\text{otherwise}
	\end{cases}
	\]
	The function $g(x)$ is continuous at precisely one point, that is 0. Now we just let the function $f$ we are looking for to be $f(x)=g(x-a)$.
\end{problem}

\begin{problem}
	\enuma{
	\item \[f(x)=
	\begin{cases}
		\frac{1}{k}, &\frac{1}{k+1}<x<\frac{1}{k} \text{for k=1,2,\dots}\\
		0, &\text{otherwise}
	\end{cases}
	\] ()Not confirmed)
	\item  $$f(x)=\prod_{k=1}^{\infty}\frac{1}{x-\frac{1}{k}}.$$ Even though infinite polynomials have not been defined I conjecture that this implies the function is not continuous at $x=0$ because $k\to \infty$.
	}
\end{problem}