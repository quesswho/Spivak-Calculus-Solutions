\chapter{Numbers of various sorts}

\begin{problem}
	\enum{
		\item The formula is clearly true for $n=1$. Suppose $$1^2+\cdots+n^2=\frac{n(n+1)(2n+1)}{6}$$ then \\
		$\begin{aligned}
			1^2+\cdots+(n+1)^2=\frac{n(n+1)(2n+1)}{6}+n^2+2n+1 \\
			\frac{2n^3+3n^2+n+6n^6+12n+6}{6}=\frac{(n+1)(n+2)(2n+3)}{6}
		\end{aligned}$
		\item The base case $n=1$ is obviously true, suppose $$(1+\cdots+n)^2=1^3+\cdots+n^3$$ 
		then \\
		$\begin{aligned}
			((1+\cdots+n)+(n+1))^2=(1+\cdots+n)^2+2(1+\cdots+n)(n+1) + (n+1)^2 \\
			=(1^3+\cdots+n^3)+n(n+1)^2+(n+1)^2=1^3+\cdots+n^3+(n+1)^3
		\end{aligned}$
	}
\end{problem}

\begin{problem}
	\enum{
	\item We solve by rewriting the sum,\\
	$\begin{aligned}
			1+3+5+\cdots+(2n-1)=1+\cdots+2n - (2+4+\cdots2n)\\
			1+\cdots+2n - 2(1+2+\cdots n)=n(2n+1)-n(n+1) \\
			=n^2 \\
		\end{aligned}$\\
	Thus $\sum_{i=1}^{n}(2i-1)=n^2$.
	\item Using similar methods as before:\\ $1^2+3^2+5^2+\cdots+(2n-1)^2=1^2+\cdots+(2n)^2-(2^2+4^2+\cdots+(2n)^2)$
	$\begin{aligned}
		1^2+\cdots+(2n)^2-4(1^2+2^2+\cdots+n^2)&=\frac{2n(2n+1)(4n+1)}{6}-\frac{4n(n+1)(2n+1)}{6}\\
		\frac{2n[8n^2+6n+1-2(2n^2+3n+1)]}{6}&=\frac{8n^3-2n}{6}
	\end{aligned}$
	We conclude that $\sum_{i=1}^{n}(2i-1)^2=\frac{8m^3-2n}{6}$
	}
\end{problem}

\begin{problem}
	\enuma{
	\item The easiest way seems to be starting at the right side.
		\begin{align*}
			\binom{n}{k-1}+\binom{n}{k}=\frac{n!}{(k-1)!(n-k+1)!}+\frac{n!}{k!(n-k)!} \\
			\frac{kn!}{k!(n-k+1)!}+\frac{(n-k+1)n!}{k!(n-k+1)!} = 	\frac{(n+1)!}{k!(n+1-k)!}=\binom{n+1}{k}
		\end{align*}
	\item We perform induction on $n$. Let the base case be $n=1$ then by the definition of the binomial coefficients we have two cases, $\binom{1}{0}$ and $\binom{1}{1}$, both of these are equal to one by definition. \\\textbf{TODO:} Is this base case actually valid? $n=1$ does not use every part of the definition. If the definition can be different for numbers other than the base case does it invalidate the proof? Must there be multiple base cases involved testing each part of the definition?
	\\\\
	Suppose $\binom{n}{k}$ is a natural number for every $0\leq k\leq n$, then it follows that $\binom{n+1}{k}=\binom{n}{k-1}+\binom{n}{k}$ is a natural number because $\binom{n}{k-1}$ and $\binom{n}{k}$ are both natural numbers.
	
	\item Suppose we can chose any number only once from $1,2,\dots,n$. First we have $n$ different choices, then $n-1$ choices and so on. If we do this $n$ times we eventually only have one number left to choose. The numbers have $n!$ different sequences from which we can choose. Now we want to count the number of integers we can choose with only $k$ choices. If $k=1$ then there is only one choice, $n$. If $k=2$ then we have $n\cdot (n-1)$, if we continue this we notice that this is the same as cutting the smaller factors in $n!$. Thus the number of ways we can do this is $\frac{n!}{(n-k)!}$. This is also known as the numbers of permutations of $n$ of length $k$. Notice that the order in which these numbers are chosen are important. However, in a set the order does in fact not matter. The number of permutations of length $k$ is $k!$, so finally we divide the permutations formula by $k$ which means that we have $\frac{n!}{k!(n-k)!}$ and by the definition of the binomial coefficients this is the same as $\binom{n}{k}$.
	Because $\binom{n}{k}$ is the number of sets with exactly $k$ integers chosen by $n$ we have that $\binom{n+1}{k}$ is the number of sets with exactly $k$ integers chosen by $n+1$. This implies that the case for $n+1$ is also a finite amount.
	
	\item Base case: $n=1$ then $(a+b)^1=\sum_{k=0}^{1}\binom{n}{k}a^{n-k}b^k =a+b$. Before we continue, a new notation must be introduced, $\sum_{0\leq k \leq n}a_k$, this means the sum over k from $1$ to $n$. Suppose $$\sum_{k=0}^{n}\binom{n}{k}a^{n-k}b^k=(a+b)^n.$$
	then 
	\begin{gather*}
		\sum_{0\leq k \leq n+1}\binom{n+1}{k}a^{n+1-k}b^k = a^{n+1}+b^{n+1}+\sum_{1\leq k\leq n}\binom{n+1}{k}a^{n-k+1}b^k \\
		a^{n+1}+\sum_{1\leq k\leq n}\binom{n}{k}a^{n-k+1}b^k+\sum_{1\leq k\leq n+1}\binom{n}{k-1}a^{n-k+1}b^k \\
		\sum_{0\leq k\leq n}\binom{n}{k}a^{n-k+1}b^k+\sum_{0\leq k-1\leq n}\binom{n}{k-1}a^{n-(k-1)}b^{(k-1)+1} \\
		\intertext{(Substitute $k-1$ for $k$)}
		a\sum_{0\leq k\leq n}\binom{n}{k}a^{n-k}b^k+b\sum_{0\leq k \leq n}\binom{n}{k}a^{n-k}b^k \\
		(a+b)\sum_{k=0}^{n}\binom{n}{k}a^{n-k}b^k=(a+b)^{n+1}
	\end{gather*}
	To conclude, we have proven the equality $$(a+b)^n=\sum_{k=0}^{n}\binom{n}{k}a^{n-k}b^n$$
	
	\item Using similar steps as above we can derive the transformation $$\sum_{k=0}^{n+1}a_k\binom{n+1}{k}=\sum_{k=0}^{n}a_k\binom{n}{k}+\sum_{k=0}^{n}a_{k+1}\binom{n}{k}$$
	
		\enum{
		\item For the sake of base case, let $n=0$, then we have $\binom{0}{0}=1$. Suppose $\sum_{j=0}^{n}\binom{n}{j}=2^n$
		then
		\begin{gather*}
			\sum_{0\leq j \leq n+1}\binom{n+1}{j}=\sum_{0\leq j \leq n}\binom{n}{j}+\sum_{0\leq j \leq n}\binom{n}{j} \\
			2\sum_{0\leq j \leq n}\binom{n}{j} =2^{n+1}
		\end{gather*}
		\item The base case $n=0$ does not seem to work, therefore we try $n=1$ then we have $\binom{n}{1}-\binom{1}{1}=1-1=0$. Suppose $\sum_{j=0}^{n}(-1)^j\binom{n}{j}=0$. Then by replacing $n$ with $n+1$ we do the same transformation as earlier
		\begin{gather*}
			\sum_{j=0}^{n+1}(-1)^j\binom{n+1}{k}=\sum_{j=0}^{n}(-1)^j\binom{n}{k}+\sum_{j=0}^{n}(-1)^{j+1}\binom{n}{k} \\
			\sum_{j=0}^{n}(-1)^j\binom{n}{k}-\sum_{j=0}^{n}(-1)^j\binom{n}{k} = 0.
		\end{gather*}
		
		\item Subtract (i) with (ii), then $$\binom{n}{0}-\binom{n}{0}+\binom{n}{1}+\binom{n}{1}+\cdots=2^n-0$$
		The end of the sum depends on whether $n$ is odd or even so we don't explicitly write it down. Notice that the all the even binomials cancel out, thus we have 
		$$2\sum_{l \text{ odd}}\binom{n}{l}=2^n \iff \sum_{l \text{ odd}}\binom{n}{l}=2^{n-1}$$
		
		\item Subtracting (i) with (iii), it follows that we only have the even binomials left, the other part of the formula is then $2^n-2^{n-1}=2^{n-1}$. It follows that $$\sum_{l \text{ even}}\binom{n}{l}=2^{n-1}.$$
		}
	}
\end{problem}

\begin{problem} % 4
	\enum{
	\item We first need to prove an important property of sums, 
	\begin{align*}
&\left(\sum_{i=0}^{n}a_i\right)\left(\sum_{j=0}^{m}b_j\right)=(a_1+\cdots+a_n)(b_1+\cdots+b_m)\\
		&=a_1(b_1+\cdots+b_m)+\cdots+ a_n(b_1+\cdots+b_m) \tag*{(Distributive property)}\\
		&=\sum_{i=0}^{n}\left(a_i\sum_{j=0}^{m} b_j\right)=\sum_{i=0}^{n}\sum_{j=0}^{m}a_ib_j \tag*{($a_i$ is a constant in the $b_j$ sum)}
	\end{align*}
	Note that we can interchange the summation symbols if we were to apply the distributive property as $b_1(a_1+\cdots+a_n)+\cdots b_m(a_1+\cdots+a_n)$ and then continue with similar steps, thus we have $$\sum_{i=0}^{n}\sum_{j=0}^{m}a_ib_j=\sum_{j=0}^{m}\sum_{i=0}^{n}a_ib_j.$$
	Now consider the polynomial $(1+x)^n(1+x)^m$, we can write this as
	\begin{gather*}                      
		\sum_{l=0}^{n+m}\binom{n+m}{l}x^l \tag{1}\\
		\intertext{and}                     
		\left(\sum_{i=0}^{n}\binom{n}{i}x^i\right)\left(\sum_{j=0}^{m}\binom{m}{j}x^j\right) = \sum_{i=0}^{n}\sum_{j=0}^{m}\binom{n}{i}\binom{m}{j}x^{i+j} \tag{2}
	\end{gather*}
	The important thing is to realize that both sums represent the same polynomial. We must now use the fact that if a polynomial is equal for any $x$ then they must have the same coefficients. This theorem can be proven as a corollary to the problem $3.7.$ (c). 
	
	The coefficients for $x^l$ is $\binom{n+m}{l}$ as stated above (1). By the theorem recently stated we know that the coefficients in the other sum must be the same. Therefore, it makes sense to gather all the coefficients to each indeterminate. To do this we will use the equality 
\begin{gather*}
	\sum_{i=0}^{n}\sum_{j=0}^{m}a_{i+j}=\sum_{l=0}^{n+m}\sum_{i=0}^{n}a_l \tag*{where $l=i+k$}
\end{gather*}
	\\
	Proof. Let $l=i+j$, then
	\begin{gather*}
		\sum_{i=0}^{n}\sum_{j=0}^{m}a_{i+j}=\sum_{0 \leq i \leq n}\sum_{0 \leq l-i \leq m}a_l \tag*{(Substitute $j=l-i$)} \\
		\sum_{0 \leq i \leq n}\sum_{0 \leq l \leq m+n}a_l=\sum_{l=0}^{m+n}\sum_{i=0}^{n}a_l \tag*{(Add $0\leq i \leq n$ to the left bound)}.
	\end{gather*}
	Applying the transformation to (2) we get the following equalities
	\begin{gather*}
		\sum_{l=0}^{n+m}\sum_{i=0}^{n}\binom{n}{i}\binom{m}{l-i}x^l=\sum_{l=0}^{n+m}\binom{n+m}{l}x^l \\
		\sum_{i=0}^{n}\binom{n}{i}\binom{m}{l-i}=\binom{n+m}{l}
	\end{gather*}
	The last equality holds by the theorem stated earlier that the if two polynomials are equal then they have the same coefficients.
	\item By letting $n=l=m$ in the equality before we get 
	\begin{gather*}
		\sum_{k=0}^{n}\binom{n}{k}\binom{n}{n-k}=\binom{2n}{n} \\
		\sum_{k=0}^{n}\binom{n}{k}^2=\binom{2n}{n}
	\end{gather*}
	The last equality holds because
	$$\binom{n}{n-k}=\frac{n!}{(n-k)!(n-(n-k))!}=\frac{n!}{(n-k)!k!}=\binom{n}{k}$$
	}
\end{problem}

\begin{problem} % 5
	We will not prove this inductively because it is trivial. Suppose $S=1+r+\cdots+r^n$ then $$rS+1=1+r+r^2+\cdots r^{n+1}=S+r^{n+1} \iff 1-r^{n+1}=S(1-r) \iff S=\frac{1-r^{n+1}}{1-r}$$
\end{problem}

