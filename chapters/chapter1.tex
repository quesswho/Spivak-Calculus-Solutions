\chapter{Basic Properties of Numbers}

\begin{problem} % 1
	\enum{
	\item Suppose that $ax=a$ and $a\neq 0$, then there exists a number $a^{-1}$. Multiplying $a^{-1}$ on both sides yields
		\begin{align*}
			(a^{-1}a) \cdot x &=a^{-1}a \\
			x &=1
		\end{align*}
	as desired.
	
	\item Applying the distributive property on $(x-y)(x+y)$ makes 
	\begin{align*}
		&(x-y)(x+y) = (x-y)x+(x-y)y\\
		&=x^2-yx+xy-y^2=x^2-y^2
	\end{align*}

	\item If we have $x^2=y^2$ then we certainly have $0=x^2-y^2$. By (ii) we have $0=(x-y)(x+y)$, this implies that $x-y=0$ or $x+y=0$, this is equivalent to saying that $x=y$ or $x=-y$.

	\item Same method as (ii):
	\begin{align*}
		&(x-y)(x^2+xy+y^2) = (x-y)x^2+(x-y)xy+(x-y)y^2\\
		&=x^3-yx^2+x^2y-xy^2+xy^2-y^3\\
		&=x^3-y^3
	\end{align*}

	\item We prove this by induction, the base case $n=2$ is already proven in (ii). Suppose $x^n-y^n=(x-y)(x^{n-1}+x^{n-2}y+\cdots +xy^{n-2}+y^{n-1})$ is true. Then we equivalently have $x^n=(x-y)(x^{n-1}+x^{n-2}y+\cdots +xy^{n-2}+y^{n-1})+y^n$. We now prove the n+1 case:
	\begin{align*}
		&x^{n+1}-y^{n+1}=x\cdot x^n-y^{n+1} \\
		&=x(x-y)(x^{n-1}+x^{n-2}y+\cdots +xy^{n-2}+y^{n-1})+xy^n-y^{n+1} \\
		&=(x-y)(x^{n}+x^{n-1}y+\cdots +x^2y^{n-2}+xy^{n-1})+(x-y)y^n \\
		&=(x-y)(x^{n}+x^{n-1}y+\cdots +xy^{n-1}+y^n)
	\end{align*}
		The resulting relation concludes the finite induction, thus $x^n-y^n=(x-y)(x^{n-1}+x^{n-2}y+\cdots +xy^{n-2}+y^{n-1})$.
		
	\item We know from (iv) that $a^3-b^3=(a-b)(a^2+ab+b^2)$, by letting $a=x$ and $b=-y$ we get $x^3+y^3=(x+y)(x^2-xy+y^2)$.
 	}
\end{problem}

\begin{problem} % 2
	Multiplying by the multiplicative inverse of $x-y$ works only when $x-y\neq 0$, that is $x\neq y$, however, the hypothesis explicitly states $x=y$. So it is not possible to find the multiplicative inverse of $x-y$ and thus the step is invalid.
\end{problem}


\begin{problem} % 3
	\enum{
	\item Say we have $\frac{a}{b}$ and $b\neq 0$ then the same fraction can be written as $ab^{-1}$. Suppose we also have a variable $c$ such that $c \neq 0$, then we have $ab^{-1} \cdot (cc^{-1})$ and consequently $(ac)(b^{-1}c^{-1})=\frac{ac}{bc}$. The final equality holds by (iii) which is proven below.
	
	\item By (i) $\frac{ad}{bd}+\frac{bc}{db} = ad(bd)^{-1}+bc(bd)^{-1}=(ad+bc)(bd)^{-1}=\frac{ad+bc}{bd}$
	
	\item $ab$ exists if $a,b \neq 0$. Let $x=(ab)^{-1}$, then
	\begin{align*}
		&x(ab)=(ab)^{-1}(ab) = (xa)b = 1 \hspace{1cm} &\text{(Multiply $x$ with $ab$)}\\
		&(xa)(bb^{-1})=b^{-1}=xa=b^{-1} &\text{(Multiply by $b^{-1})$} \\
		&x(aa^{-1})=b^{-1}a^{-1} = x& \text{(Multiply by $a^{-1}$)}
	\end{align*}
	
	\item Suppose $b,d\neq 0$, then $\frac{a}{b} \cdot \frac{c}{d}=(ab^{-1}) \cdot (cd^{-1}) =(ac)(b^{-1}d^{-1})=(ac)(bd)^{-1}=\frac{ac}{bd}$

	\item Suppose $b,c,d \neq 0$, then $\frac{a}{b} \big/ \frac{c}{d}=(ab^{-1})(cd^{-1})^{-1}=(ab^{-1})(c^{-1}d)=(ac)(bd)^{-1}=\frac{ac}{bd}$
	
	\item Suppose $b,d \neq 0$. Assume $\frac{a}{b}=\frac{c}{d}$, multiplying by $bd$ on both side yields the relation $ad=bc$. For the converse multiply $ad=bc$ by $(bd)^{-1}$.
	}
\end{problem}

\begin{problem} % 4
	\enum{
	\item $4-x < 3-2x \iff (4-4)+(-x+2x) < (3-4)+(2x-2x) \iff x < -1$.
	
	\item $5-x^2<8 \iff -3 < x^2$. Note that $x^2 \geq 0$ and for every single value of $x$, so our solution is every $x$.
	
	\item $5-x^2 < -2 \iff 7 < x^2 \iff \sqrt{7} < x$ or $-\sqrt{7} > x $.

	\item The product is positive when $x-1 > 0$ and $x-3>0$ or when $x-1 < 0$ and $x-3<0$, that is when $x > 3$ or when $x<1$.
	
	\item Complete the square $x^2-2x+2=(x-1)^2+1$. The product $(x-1)^2$ is always positive, and since we have the $+1$ as well in the inequality, this inequality must be true for every single $x$.
	
	\item The inequality is equivalent to $x^2+x-1 > 0$. Completing the square $(x+\frac{1}{2})^2 > \frac{5}{4}$. If $x \geq -\frac{1}{2}$ then $x > \frac{-1+\sqrt{5}}{2}$. If $x < -\frac{1}{2}$ then $x < \frac{-1-\sqrt{5}}{2}$. Thus, the solution is $x > \frac{-1+\sqrt{5}}{2}$ and $x <\frac{-1-\sqrt{5}}{2}$.
	
	\item Equivalently we have $(x-\frac{1}{2})^2>\frac{25}{4}$. If $x\geq \frac{1}{2}$ then $x > 3$ if $x < \frac{1}{2}$ then $x < -2$. The solution set is $x >3$ and $x <-2$.
	
	\item Equivalently $(x+\frac{1}{2})^2+\frac{3}{4} > 0$. This is true for every x because $(x+\frac{1}{2}) \geq$ and $\frac{3}{4}> 0$. Adding them gives $(x+\frac{1}{2})^2 +\frac{3}{4} > 0$.
	
	\item Let $b=(x+5)(x-3)$. Then $b$ is positive if $x > 3$ or $x < -5$ and negative if $-5<x<3$. Let $a=x-\pi$. $a$ is positive if $x > \pi$. $ab$ is positive if both $a$ and $b$ are positive or if both are negative. So $ab$ is positive if $x > \pi$ ($b$ must be positive because $x > 3$). $ab$ is negative if $-5<x<3$ (This implies $x<\pi$).
	
	\item If $x > \sqrt[3]{2}$ and $x > \sqrt{2}$ then the product is positive, thus the first solution is $x > \sqrt{2}$. If $x < \sqrt[3]{2}$ and $x < \sqrt{2}$ then the product is positive. The second solution is $x < \sqrt[3]{2}$.
	
	\item Apply $\log_2$ on both sides: $x < 3$.
	
	\item Suppose $x < 1$, we will show this is a solution. We have $3^x<3^1 = 3$, adding $x<1$ to the inequality we get $x+3^x<3+1=4$. Since both $3^x$ and $x$ are strictly increasing expressions finding the inequality $x<1$ suffices as all real solutions.
	
	\item Noting that $x\neq 0$ and $x\neq 1$. Expanding the fractions we get $\frac{1-x}{x(1-x))}+\frac{x}{x(1-x)}=\frac{1}{x(1-x)}>0$. The solutions depends on if the denominator is positive. Thus $x(1-x)>0$ has the same solution set. The solutions are $0 < x < 1$.
	
	\item Note $x\neq -1$. Expand by $(x+1)$: $\frac{(x-1)(x+1)}{(x+1)^2}>0$. Since the denominator is always positive we can multiply this on both sides, $x^2-1>0$, Thus $x < -1$ and $x > 1$.
	}
\end{problem}

\begin{problem} % 5
	\enum{
	\item Suppose $a<b$ and $c<d$ then we have $b-a>0$ and $d-c >0$ by property 11 $(b-a)+(d-c) > 0$ which is the same as $b+d>a+c$.
	
	\item Suppose $a < b$ then $0 < b-a \iff -b < (b-b)-a = -b < -a$.
	
	\item Suppose $a < b$ and $c < d$, by (ii): $-c < -d$, then by (i) we have $a-d < b-d$.
	
	\item Suppose $a < b$ then $b-a > 0$. Assume $c>0$, Using (P12) we know that $c(b-a) > 0$ and consequently $bc-ac > 0 \iff bc > ac$.
	
	\item Suppose $a < b$ then $b-a > 0$. Assume $c<0$, then by (ii) we have $-c>0$. Using P12 we know that $-c(b-a) > 0$ and consequently $ac-bc > 0 \iff ac > bc$.
	
	\item Since $a > 1 > 0$ we apply (iv) by letting $c=a$. Thus $a^2>a$.
	
	\item Because $a$ is positive, it follows by applying (iv) to $a < 1$ that $a^2 <a$.
	
	\item Using (iv), multiply $a<b$ with $c$ and $c<d$ with $b$. This means that we have $ac<bc$ and $bc<bd$, this is the same as $ac<bc<bd$, thus $ac <bd$.
	
	\item Using (viii) we multiply the same inequality twice, $a^2<b^2$.
	
	\item Suppose $a,b\geq 0$, we prove the contra-positive, therefore $a\geq b$. Multiply by $a$ and $b$ respectively gives two inequalities $a^2 \geq ab$ and $ab \geq b^2$ which is the same as $a^2\geq ab \geq b^2$. This concludes the contra-positive proof because $a^2 \geq b^2$ is the logical opposite of $a^2 < b^2$.
}
\end{problem}

\begin{problem} % 6
	\enuma{
		\item The base case is $n=2$ which was proven in problem 1.5. Assume $x^n<y^n$ for $0 \leq x < y$. By problem 1.5. (viii) we have $x \cdot x^n < y \cdot y^n \iff x^{n+1}<y^{n+1}$. The induction is complete, thus if $0 \leq x,y$ then $x^n<y^n$ for $n=1,2,\ldots$.
		
		\item Suppose $x<y$ and $n=2k+1$, We have three cases.
		\enum{
			\item $x,y \geq 0$, this case has been proven in (a).
			\item $x \leq 0$ and $y \geq 0$. Consider $x^n$, because $n$ is odd, it has the following property, $x^{2k+1}=x \cdot (x^k)^2 < 0$, because $x$ is negative and $(x^k)^2$ is positive. However $y^n>0$ because $y$ is positive. This means we have $x^n<0<y^n$.
			
			\item $x,y < 0$, by the inequality we have $-x > 0$ and $-y > 0$. We also have $-y < -x$, by (a) we have $(-y)^n<(-x)^n \iff -y^n < -x^n$ because n is odd. Finally we have $x^n < y^n$.
		}
	
		\item Suppose $x^n=y^n \iff x^n-y^n=0 = (x-y)(x^{n-1}+x^{n-2}y+ \cdots + xy^{n-2}+y^{n-1})$ Then either $x-y=0$ or $x^{n-1}+x^{n-2}y+ \cdots + xy^{n-2}+y^{n-1}=0$ In the first case $x=y$, in the second case we first note that $x^n=y^n$ implies that $x$ and $y$ has the same sign and thus $x^{n-1}+x^{n-2}y+ \cdots + xy^{n-2}+y^{n-1} \geq 0$ where the equality holds only when $x,y=0$ then $x=y$ is still true. 
		
		\item Let $n$ be an even positive integer. Next we prove the contra-positive, suppose $|x| \neq |y|$ ($x=y$ or $x=-y$ is the same as saying $|x|=|y|$). Consequently this means either $|x|<|y|$ or $|x|>|y|$. By (a) this means that either $|x|^n<|y|^n$ or $|x|^n>|y|^n$. Because $n$ is even this is equivalent to $x^n<y^n$ or $x^n>y^n$ which is the logical complement of $x^n=y^n$.
	}
\end{problem}

\begin{problem} % 7
	Suppose $0<a<b$, multiply by $a$ then $a^2<ab \iff a < \sqrt{ab}$. Next consider $(a-b)^2 > 0$ which is equivalent to $a^2+b^2 +2ab > 4ab \iff \frac{a+b}{2} > \sqrt{ab}$, this means that we have $a<\sqrt{ab}<\frac{a+b}{2}$ now remains the final inequality. By the premise we have $a-b<0 \iff a+b<2b \iff \frac{a+b}{2}<b$. We conclude by stating $a<\sqrt{ab}<\frac{a+b}{2}<b$.
\end{problem}

\begin{hproblem} % 8
	\begin{itemize}
		\item [(P10)] Let $b=0$ in P'10, then for every $a$ one of the following properties apply 
		\enum{
			\item $a=0$
			\item $a<0$
			\item $a>0$
		}
		Because the collection $P$ contains all the numbers $x$ such that $x>0$, we can see that (iii) states that $a$ belongs to $P$. (ii) is equivalent to $-a>0$, thus -a is  in $P$.
		\item [(P11)] Suppose $x$ and $y$ are in $P$ then $0<x$ and $0<y$. By P'12 (Let a=0) we have $x<y+x$. By P'11  we get $0<y+x$ which is in $P$.
		\item [(P12)] Suppose $x$ and $y$ are in $P$ then $0<x$ and $0<y$. Using P'13 we get $0<xy$, this means that $xy$ is in $P$.
	\end{itemize}
\end{hproblem}

\begin{problem} % 9
	\enum{
	\item $\sqrt{2}+\sqrt{3}-\sqrt{5}+\sqrt{7}$.
	\item Triangle inequality states that $|a+b|-|a|-|b|\leq 0$. Therefore $|a|+|b|-|a+b|$.
	\item Triangle inequality gives $|(a+b)+c| -|a+b|-|c| \leq 0 \iff |a+b|+|c|-|a+b+c|. \geq 0$. Our solution is therefore $|a+b|+|c|-|a+b+c|$.
	\item $x^2-2xy+y^2=(x-y)^2\geq 0$, thus $x^2-2xy+y^2$.
	\item $\sqrt{2}+\sqrt{3}+\sqrt{5}-\sqrt{7}$
	}
\end{problem}

\begin{problem} % 10
	\enum{
		\item Suppose $a+b\geq 0$ and $b\geq 0$ then we have $a+b-b=a$. Suppose $a+b\geq 0$ and $b<0$ then $a+b+b=a+2b$. Suppose $a+b<0$ and $b\geq 0$ then $-a-b-b=-(a+2b)$. Suppose $a+b<0$ and $b<0$ then $-a-b+b=-a$.
		
		\item If $0\geq x \geq 1$ then $1-x$. If $-1\geq x < 0$ then $1+x$. If $1<x$ then $x-1$ then $-x-1$.
		
		\item If $x \geq 0$ then $x-x^2$, if $x < 0$ then $-x-x^2$.
		
		\item If $a \geq 0$ then $a$, if $a < 0$ then $3a$.
	}
\end{problem}

\begin{problem} % 11
	\enum{
	\item Suppose $x-3>0$ then $x-3=8 \iff x = 11$. Suppose $x-3<0$ then $3-x=8 \iff x=-5$.
	
	\item Suppose $x-3 \geq 0$ then $3 \leq x <11$. Suppose $x-3<0$ then $-5 < x < 3$. Combining both inequalities $-5 < x < 11$.
	
	\item Suppose $x+4 \geq 0$ then $x < -2$, so $-4 \leq x<-2$. If $x+4 < 0$ then $-6<x<-4$. Combining both inequalities gives $-6<x<-2$.
	
	\item Suppose $x \leq 2$ then $x-1+x-2>1 \iff x>2$. This means $x>2$ is always a solution. Suppose $1\leq x < 2$, then $x-1-x+2 > 1 \iff 1>1$, which can not be true. Suppose $x < 1$, then $1-x-x+2>1 \iff x<1$. The solution is $x < 1$ and $x>2$.
	
	\item Suppose $x \geq 1$ then $x-1+x+1<2 \iff x < 1$ which is a contradiction. Suppose $-1 \leq x < 1$ then $1-x+x+1<2 \iff 2<2$, also contradiction. Suppose $ x < -1$ then $1-x-x-1<2 \iff x > -1$, an $x$ that satisfies the inequality is nonexistent.
	
	\item Suppose $x \geq 1$ then $x-1+x+1<1 \iff x < \frac{1}{2}$ which is a contradiction. Suppose $-1 \leq x < 1$ then $1-x+x+1<1 \iff 2<1$,also a contradiction. Suppose $ x < -1$ then $1-x-x-1<1 \iff x > -\frac{1}{2}$, similarly to (iv), there are no $x$ that satisfy the inequality.
	
	\item We have $x-1=0\iff x=1$ or $x+1=0 \iff x=-1$.
	
	\item  Suppose $x \geq 1$ then $(x-1)(x+2)=3 \iff x^2+x-5=0 \iff (x+\frac{1}{2})^2=\frac{21}{4} \implies x = \frac{-1+\sqrt{21}}{2}$. Suppose $-2\leq x<1$ then $(1-x)(x+2)=3$ which is a polynomial with complex roots thus no solutions there. Suppose $x<-2$, then we get the same polynomial as in the first case because $(-1)^2=1$, so the other root is $x=\frac{-1-\sqrt{21}}{2}$ which is less than $-2$ because $\frac{-1-\sqrt{21}}{2}<\frac{-1-\sqrt{16}}{2}=\frac{-5}{2}<-2$. To conclude $x=\frac{-1 \pm \sqrt{21}}{2}$
	}
\end{problem}

\begin{problem} % 12
	\enum{
	\item $|xy|^2=(xy)^2=x^2y^2=|x|^2|y|^2\iff |xy|=|x|\cdot|y|$
	\item Consider $\left| \frac{1}{x} \right|$ for $x\neq 0$. This is the same as $\sqrt{(\frac{1}{x})^2}=\sqrt{\frac{1}{x^2}}=\frac{1}{\sqrt{x^2}}=\frac{1}{|x|}$.
	
	\item Suppose $y\neq 0$ then $\left| \frac{x}{y} \right| = \sqrt{(\frac{x}{y})^2}=\frac{\sqrt{x^2}}{\sqrt{y^2}}=\frac{|x|}{|y|}$
	
	\item Suppose $a,b$ are real numbers, then the triangle inequality is $|a+b|\leq |a|+|b|$. Let $a=x$ and $b=-y$ then $|x-y| \leq |x|+|-y|=|x|+|y|$. The final equality is proven by $|-y|=\sqrt{(-y)^2}=\sqrt{(-1)^2y^2}=\sqrt{y^2}$.
	
	\item Using the triangle inequality $|x|=|(x-y)+y| \leq |x-y|+|y| \iff |x|-|y| \leq |x-y|$
	
	\item There are two cases from the inequality, $|x|-|y| \leq |x-y|$ and $|y|-|x| \leq |y-x|$, note that the last absolute value comes from the fact $|x-y|=|y-x|$. Both inequalities are identical to (v) (the second inequality has the variables interchanged).
	
	\item We have $|(x+y)+z| \leq |x+y|+|z| \leq |x|+|y|+|z|$. Doing the case work for the equality is left to the reader.
	}
\end{problem}

\begin{problem} % 13
	We start by proving for $\max$, let $x \geq y$ then $\max(x,y)=\frac{x+y+x-y}{2}=x$ Likewise if $y \geq x$ then $\max(x,y)=y$. Similar reasoning shows that the formula for $\min(x,y)$ is valid. Next we use substitution and get $\max(x,y,z)=\max(x, \max(y,z))=\frac{y+z+2x+|y-z|+|y+z+2x+|y-z||}{4}$ and $\min(x,y,z)=\min(x, \min(y,z))=\frac{y+z+2x+|y-z|-|y+z+2x+|y-z||}{4}$.
\end{problem}

\begin{problem} % 14
	\enuma{
	\item Suppose $a \geq 0$ then we have $a=-(-a)$. The case for $a\leq 0$ is then obvious because we have $(-a)\geq 0$ which can be used on the previously proven fact.
	
	\item ($\Rightarrow$) Suppose $-b\leq a \leq b$, this implies $a \leq b$ and $-b\leq a \iff -a \leq b$ and consequently $|a| \leq b$. \\($\Leftarrow$) Suppose $|a| \leq b$ then $a \leq b$ and $-a \leq b \iff -b \leq a$, thus $-b \leq a \leq b$. Now we prove the last part. Suppose $|a| \leq |a|$ then by the previously proven theorem we have $-|a| \leq a \leq |a|$.
	
	\item As proven earlier, for every $a,b$ we have $-|a| \leq a \leq |a|$ and  $-|b| \leq b \leq |b|$. Add these together gives $-(|a|+|b|) \leq a+b \leq |a|+|b|$, applying the theorem from (b) on $(|a|+|b|)$ and $(a+b)$ we get $|a+b| \leq |a|+|b|$.

	}
\end{problem}

\begin{hproblem} % 15
	We prove first that if $x=y$ and $x,y\neq 0$. The inequality is then $x^2+x^2+x^2>0 \iff x^2 >0$ which is true because $x\neq 0$.
	
	Suppose $x\neq y$, then the left side of inequality is equivalent to $(x^2+xy+y^2)=\frac{x^3-y^3}{(x-y)}$. Suppose $x>y$ then $x^3-y^3>0$ by problem 6 (b),  since both the numerator and denominator are positive we know that $\frac{x^3-y^3}{(x-y)}>0$. Next we assume $x < y$ which implies $x^3-y^3 < 0$ by problem 6 (b). This means the numerator and denominator are both negative, thus $\frac{x^3-y^3}{(x-y)}>0$. In every case the inequality is positive, thus we have proven that $x^2+xy+y^2>0$.
	
	To prove that the second inequality holds we follow the same steps, suppose $x=y$ which means the inequality is $5x^4>0$. Next suppose $x\neq y$ then we have $x^4+x^3y+x^2y^2+xy^3+y^4=\frac{x^5-y^5}{x-y}$. Suppose $x-y>0$ then $x^5-y^5>0$ which implies $\frac{x^5-y^5}{x-y}>0$. Assume $x-y<0$ then $x^5-y^5<0$ which implies $\frac{x^5-y^5}{x-y}>0$.
\end{hproblem}

\begin{hproblem} % 16
	\enuma{
	\item $(x+y)^2=x^2+2xy+y^2=x^2+y^2 \iff xy=0$ which implies $x=0$ or $y=0$. Next we have $(x+y)^3=x^3+3x^2y+3xy^2+y^3=x^3+y^3 \iff x^2y+xy^2=0=xy(x+y)$. Which implies either $x=0$ or $y=0$ or $x=-y$.
	
	\item Consider $3(x+y)^2=3x^2+6xy+3y^3 \geq 0$, since $x,y \neq 0$ we have $x^2 > 0$ and $y^2>0$, adding these inequalities makes $4x^2+6xy+4y^2 > 0$. If $x,y =0$ then the statement would be false.
	
	\item Equivalently we have $4x^3y+6x^2y^2+4y^3x=xy(4x^2+6xy+4y^2)$, left side indicates that it is equal to zero when $x=0$ or $y=0$. Thus $(x+y)^4=x^4+y^4$ when $x=0$ or $y=0$.
	
	\item Subtract with $x^5+y^5$ and since $xy \neq 0$ we divide by $5xy$ this makes $x^3+2x^2y+2xy^2+y^3=0 \iff (x+y)^3=x^2y+y^2x=xy(x+y)$. Suppose $x+y \neq 0$ then $xy=(x+y)^2 \iff x^2+xy+y^2=0$, this implies $x,y=0$ by letting $p=x^2+xy+y^2 \iff 2p = 2x^2+2xy+2y^2=x^2+y^2+(x+y)^2$, it then follows all the terms have to be zero because they are either zero or positive, $x=0$ and $y=0$, this contradicts the fact that $xy=0$, thus it must be the case that $x=-y$.
	
	Assume this time that $x = 0$ then $(x+y)^5=x^5+y^5 = x^5+5x^4y+10x^3y^2+10x^2y^3+5xy^4+y^5 \iff y^5 = y^5$. By interchanging $x$ with $y$ in the last sentence it follows that $x=0$ or $y=0$. To conclude, the solutions are $x=-y$ or $x=0$ or $y=0$. My guess is that the same solutions apply to $(x+y)^n=x^n+y^n$ if $n$ is odd and $x=0$ or $y=0$ if $n$ is even.
	}
\end{hproblem}

\begin{problem} % 17
	\enuma{
	\item $2x^2-3x+4=2(x-\frac{3}{4})^2+y \implies y=\frac{32}{8} - \frac{9}{8}=\frac{23}{8}$
	
	\item Subtract $2(y+1)^2$ this makes $x^2-3x$. Let $x^2-3x=(x-\frac{3}{2})+z$ then $z=-\frac{9}{4}$, $z$ is the smallest value.
	
	\item Let $m$ be the minimum number for a simple second degree polynomial, then it follows that $x^2+bx+c=0=(x+\frac{b}{2})^2+m = x^2+bx+\frac{b^2}{4}+m \iff m = c-\frac{b^2}{4}$
	
	We have $x^2+4xy+5y^2-4x-6y+7=x^2+(4y-4)x+5y^2-6y+7$ The minimum is thus $m=5y^2-6y+7-4(y^2-2y+1)=y^2+2y+3=(y+1)^2+2$. This implies that $2$ is in fact the minimum value.
}
\end{problem}

\begin{problem} % 18
	\enuma {
	\item $x=\frac{-b \pm \sqrt{b^2-4c}}{2} \iff (2x+b)^2=b^2-4c \iff 4x^2+4xb+b^2-b^2+4c=0 \iff x^2+bx+c=0$.
	\item We complete the square, $x^2+bx+c=0 \iff 4(x+\frac{b}{2})^2=b^2-4c$ this follows that $(x+\frac{b}{2})^2 \geq 0$, but $b^2-4c < 0$ which is a contradiction. It also follows that $x^2+bx+c>0$ which means there are no real values of $x$ that satisfy the equation.
	
	\item We complete the square $(x+\frac{y}{2})^2+\frac{3y^2}{4}$. Since $\frac{3y^2}{4} > 0$ because $y\neq0$ it must be the case that $(x+\frac{y}{2})^2+\frac{3y^2}{4} >0$ which is the same as $x^2+xy+y^2 > 0$
	
	\item Completing the square makes $(x+\frac{\alpha y}{2})^2+y^2(1-\frac{\alpha^2}{4})$. The left term has the property $(x+\frac{\alpha y}{2})^2\geq 0$ (just let $x=-\frac{\alpha y}{2}$). This means the right term must be positive. Let $1-\frac{\alpha^2}{4}>0$ which implies $-2<\alpha<2$.
	
	\item $ax^2+bx+c=a(x^2+\frac{bx}{a})+c=a(x+\frac{b}{2a})^2+c-\frac{b^2}{4a^2}$. Since $a>0$ the minimum must be when $x+\frac{b}{a}=0$, so the minimum is $c-\left( \frac{b}{2a} \right)^2$. (The first case is just $a=1$)
	}
\end{problem}

\begin{problem} % 19
	\enuma{
	\item Suppose $x_1=\lambda y_1$ and $x_2=\lambda y_2$ then the equality holds if $\lambda ({y_1}^2 + {y_2}^2) = \sqrt{\lambda^2({y_1}^2+{y_2}^2)}\sqrt{({y_1}^2+{y_2}^2)} \iff \lambda = |\lambda|$. Seems to be some kind of error (edition 3) because it does not hold if $\lambda$ is negative. Let's assume $\lambda \geq 0$. The then equality holds. The equality also holds if $y_1=y_2=0$ because both factors on both sides are equal to zero.
	
	Assume $y_1$ and $y_2$ is not equal to zero. Then there does not exist a $\lambda$ such that $x_1=\lambda y_1$ and $x_2=\lambda y_2$, the problems states that this implies $\lambda^2({y_1}^2+{y_2}^2)-2\lambda (x_1y_1+x_2y_2)+({x_1}^2+{x_2}^2)>0$. This equation is of the form $\lambda^2+b\lambda+c>0$ and since there does not exist any $\lambda$ we have $b^2<4ac$ by noticing that dividing by $a$ in the equation $ax^2+bx+c=0$ you can apply problem 18 (b), that is $(x_1y_1+x_2y_2)^2<({y_1}^2+{y_2}^2)({x_1}^2+{x_2}^2)$. This follows that $|x_1y_1+x_2y_2| < \sqrt{{y_1}^2+{y_2}^2}\sqrt{{x_1}^2+{x_2}^2}$
	
	To conclude we have $$x_1y_1+x_2y_2\leq  |x_1y_1+x_2y_2| \leq \sqrt{{y_1}^2+{y_2}^2}\sqrt{{x_1}^2+{x_2}^2}.$$
	
	\item We start with $(x-y)^2\geq 0 \iff 2xy \leq x^2+y^2$. Suppose $x_1,x_2,y_1,y_2 \neq 0$ and let $x=\frac{x_i}{\sqrt{{x_1}^2+{x_2}^2}}$,  $y=\frac{y_i}{\sqrt{{y_1}^2+{y_2}^2}}$ for $i=1,2$. It follows that
	\begin{gather*}
		\begin{dcases}
			\cfrac{2x_1y_1}{\sqrt{{x_1}^2+{x_2}^2}\sqrt{{y_1}^2+{y_2}^2}} \leq \cfrac{{x_1}^2}{{x_1}^2+{x_2}^2}+\cfrac{{y_1}^2}{{y_1}^2+{y_2}^2}
			\\
			\cfrac{2x_2y_2}{\sqrt{{x_1}^2+{x_2}^2}\sqrt{{y_1}^2+{y_2}^2}} \leq \cfrac{{x_2}^2}{{x_1}^2+{x_2}^2}+\cfrac{{y_2}^2}{{y_1}^2+{y_2}^2}
		\end{dcases}
	\end{gather*}
	
	Add both inequalities together, then it follows that $x_1y_1 + x_2y_2\leq \sqrt{{x_1}^2+{x_2}^2}\sqrt{{y_1}^2+{y_2}^2}$.
	
	If we assume $x_i=0$ or $y_i = 0$ for $i=1,2$ then either all the terms will be zero or the resulting inequality is for example $0 \leq {y_1}^2$ (let $x_1=0$).

	\item $({x_1}^2+{x_2}^2)({y_1}^2+{y_2}^2)$
	\begin{align*}
		&={(x_1y_1)}^2+2(x_1y_1)(x_2y_2)+{(x_2y_2)}^2+{(x_2y_1)}^2-2(x_2y_1)(x_1y_2)+{(x_1y_2)}^2 \\
		&=(x_1y_1+x_2y_2)^2+(x_2y_1-x_1y_2)^2 \geq (x_1y_1+x_2y_2)^2 \\
		&\iff \sqrt{{x_1}^2+{x_2}^2}\sqrt{{y_1}^2+{y_2}^2} \geq |x_1y_1+x_2y_2| \geq x_1y_1+x_2y_2
	\end{align*}

	\item The problem is constructed to waste time, see (a) where we already proved it. It shows that if $y_1=0$ and $y_2=0$ or there exists a number $\lambda$ such that $x_1=\lambda y_1$ and $x_2=\lambda y_2$ then the equality holds, otherwise $|x_1y_1+x_2y_2| < \sqrt{{y_1}^2+{y_2}^2}\sqrt{{x_1}^2+{x_2}^2}$.
	}

	
\end{problem}

\begin{problem}
	Add both inequalities, $|x-x_0|+|y-y_0| < \varepsilon$. We apply the triangle inequality which makes $|(x+y)-(x_0+y_0)|\leq |x-x_0|+|y-y_0|<\varepsilon$. For the second inequality, notice that that  $|y-y_0|=|y_0-y|$. So the triangle inequality makes $|(x-y)-(x_0-y_0)|\leq |x-x_0|+|y_0-y| < \varepsilon$.
\end{problem}

\begin{hproblem}
	Suppose $|x-x_0|< \frac{\varepsilon}{2(|y_0|+1)}$, then $2|x-x_0|(|y_0|+1) < \varepsilon$. Now assume $|y-y_0|< \frac{\varepsilon}{2(|y_0|+1)}$ then $2|y-y_0|(|x_0|+1) < \varepsilon$. Sum the two similar inequalities \\
	\begin{align*}
		&2|x-x_0|(|y_0|+1)+2|y-y_0|(|x_0|+1) < 2\varepsilon \\
		&|x-x_0|(|y_0|+1)+|y-y_0|(|x_0|+1) < \varepsilon \\
	&|y_0||x-x_0|+|x-x_0|+|x_0||y-y_0|+|y-y_0| < \varepsilon
	\end{align*}
	Now suppose $|x-x_0| < 1$ then we have $|y-y_0||x-x_0| < |y-y_0|$.
	Continuing on the expression above we get
	\begin{align*}
		&>|y_0||x-x_0|+|x_0||y-y_0|+|y-y_0| \\
	&>(|y_0|+|y-y_0|)(|x-x_0|)+|x_0||y-y_0| \\
	&\geq |y||x-x_0|+|x_0||y-y_0| \geq |xy-x_0y+x_0y-x_0y_0| = |xy-x_0y_0|
	\end{align*}
	Therefore we have $|xy-x_0y_0| < \varepsilon$. 
\end{hproblem}

\begin{hproblem}
	We first prove that $y\neq 0$. Suppose $|y-y_0|<\frac{|y_0|}{2}$ then by problem 12, we get $|y_0| < 2|y|$ by problem 12. By supposing $y=0$ we get a contradiction because $0<|y_0|$ thus it must be the case that $y\neq 0$. \\\\
	Now we prove the latter. Suppose $|y-y_0| < \frac{\varepsilon|y_0|^2}{2}$. Then 
	\begin{align*}
		|y-y_0| &< \varepsilon|y_0||y| \\
		\left| \frac{y_0-y}{y_0y} \right| &< \varepsilon \\
		\left| \frac{1}{y_0}-\frac{1}{y} \right| &< \varepsilon
	\end{align*}
	as desired.
\end{hproblem}

\begin{hproblem}
	We begin first by using problem 21. We can then state that if $y\neq0$, $y_0\neq0$, $\left| \frac{1}{y}-\frac{1}{y_0}\right| < \frac{\varepsilon}{2(|x_0|+1)}$ and $|x-x_0|<\min \left( \frac{\varepsilon}{2(\left| \frac{1}{y_0}\right| +1)},1 \right)$ then we have $\left| \frac{x}{y}-\frac{x_0}{y_0}\right| < \varepsilon$. Now we need to modify the hypothesis. We have that $y_0\neq 0$ and $|y-y_0| < \min \left( \frac{|y_0|}{2},\frac{\varepsilon|y_0|^2}{2} \right)$ implies $y\neq 0$ and the hypothesis earlier.\\\\
	To conclude, $y_0=0$, $|y-y_0| < \min \left( \frac{|y_0|}{2}, \frac{\varepsilon|y_0|^2}{2} \right)$ and $|x-x_0|<\min \left( \frac{\varepsilon}{2(\left| \frac{1}{y_0}\right| +1)},1 \right)$ implies $y\neq0$ and $\left| \frac{x}{y}-\frac{x_0}{y_0}\right| < \varepsilon$.
\end{hproblem}

\begin{hproblem}
	\enuma{
	\item We prove the base case (k=2) with the associative law, $(a_1+a_2)+a_3=a_1+(a_2+a_3)$. Next we suppose $P(k)$: $(a_1+\cdots+a_k)+a_{k+1}=a_1+\cdots+a_{k+1}$, then we prove for $P(k+1):$ \\\\
	$\begin{aligned}
		(a_1+\cdots+a_{k+1})+a_{k+2} &=[(a_1+\cdots+a_k) + a_{k+1}]+a_{k+2} \\
		(a_1+\cdots+a_k)+(a_{k+1}+a_{k+2})&=a_1+\cdots+k_{k+2}
	\end{aligned}$\\\\
	This concludes the induction.
	
	\item We will prove this by induction on $n$, suppose $n \geq k$ and $(a_1+\cdots+a_k)+(a_{k+1}+\cdots+a_n)=a_1+\cdots+a_n$. The base case is $n=k+1$ which was proven in the previous problem. We will now show the equality holds for $n+1$, we have \\
	$\begin{aligned}
		&(a_1+\cdots+a_k)+(a_{k+1}+\cdots+a_{n+1}) \\
		&{}= (a_1+\cdots+a_k)+((a_{k+1}+\cdots+a_n)+a_{n+1}) \\\
		&{}= ((a_1+\cdots+a_k)+(a_{k+1}+\cdots+a_n))+a_{n+1} \\
		&{}= (a_1+\cdots+a_n)+a_{n+1} \\
		&{}= a_1+\cdots+a_{n+1}
	\end{aligned}$\\\\
	We have now proven that for $n \geq k$ it follows that $$(a_1+\cdots+a_k)+(a_{k+1}+\cdots+a_n)=a_1+\cdots+a_n.$$
	
	\item We will show that $s(a_1,\dots,a_k)=s(a_1)+\cdots+s(a_k)$ by induction on k. Let the base case be $k=1$, then we obviously have an equality. Now we assume $s(a_1,\dots,a_k)=s(a_1)+\cdots+s(a_k)$ and now prove for the $k+1$ case. \\ $\begin{aligned}
		s(a_1,\dots,a_{k+1})&{}=s(a_1,\dots,a_k)+s(a_{k+1}) \\
		&{}=s(a_1)+\cdots+s(a_{k+1})
	\end{aligned}$ \\\\
	Because $s(a_1)+\cdots+s(a_k)=a_1+\cdots+a_k$, our proof is done.
	}
\end{hproblem}

\begin{problem}
	We suppose the rules of addition and multiplication given in the problem we then prove it is a field.
	\enum{
	\item Testing each case is tedious and will not be contained here, but we find that $a+(b+c)=(a+b)+c$ works.
	\item Suppose $a=0$ then $0+0=0+0=0$, and $a=1$ implies $1+0=0+1=0$
	\item If $a=0$ then then let $-a=0$ and if $a=1$ then $-a=1$.
	\item This works by exhaustion.
	\item If at least one variable is zero, then $0=0$, otherwise $1\cdot(1\cdot)=(1\cdot 1)\cdot 1 \iff 1=1$
	\item Suppose $a=0$ then $1\cdot 0=1\cdot 0=0$, suppose $a=1$ then $1\cdot1=1\cdot1=1$
	\item $a=0$ is not allowed so we only prove for the $a=1$ case which makes $a^{-1}=1$.
	\item If at least one variable is equal to zero then we have $0=0$, otherwise $1\cdot1=1\cdot1$
	\item Suppose $a=0$ then $0\cdot(b+c)=0\cdot b+0\cdot c=0$. Suppose $a=1$ then $1\cdot(b+c)=1\cdot b + 1\cdot c= b+c$
	}
\end{problem}